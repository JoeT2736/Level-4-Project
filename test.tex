%%%%% Document Setup %%%%%%%%

\documentclass[10pt,twocolumn]{revtex4-2}    % Font size (10,11 or 12pt) and column number (one or two).


\usepackage{times}                          % Times New Roman font type

\usepackage[a4paper, left=1.85cm, right=1.85cm,top=1.85cm, bottom=1.85cm]{geometry}       % Defines paper size and margin length

\usepackage[font=small, labelfont=bf]{caption}                      % Defines caption font size as 9pt and caption title bolded


\usepackage{graphics,graphicx,epsfig,ulem}	% Makes sure all graphics works
\usepackage{amsmath} 						% Adds mathematical features for equations

\usepackage{etoolbox}                       % Customise date to preferred format

\makeatletter
\patchcmd{\frontmatter@RRAP@format}{(}{}{}{}
\patchcmd{\frontmatter@RRAP@format}{)}{}{}{}
\renewcommand\Dated@name{}
\makeatother

\usepackage{fancyhdr}


\pagestyle{fancy}                           % Insert header
\renewcommand{\headrulewidth}{0pt}
\lhead{A. Student}                          % Your name
\rhead{Example Lab Report Title}            % Your report title               

\def\bibsection{\section*{References}}        % Position reference section correctly


%%%%% Document %%%%%    
\begin{document}


\title{Example Lab Report} 
\date{Submitted: \today{}, Date of Experiment: EXPERIMENT DATE}
\author{A. Student (and L. Partner)}
\affiliation{\normalfont L1 Discovery Labs, Lab Group XXX, Lab Day}


\begin{abstract}              
 
Lorem

\end{abstract}

\maketitle

\thispagestyle{plain} % produces page number for front page

Week 1
\section{Notes}

\subsection{Collective motion in biological systems}

An important reason for the interest in collective motion is the appearance of new, extremely efficient and informative techniques to collect data about the details of motions within a collective of organisms.

A transition occurs from disorder to ordered motion (flocking), as a function of the relevant parameters. 

Thus, if a system is made of units

— that are rather similar;
— moving with a nearly constant absolute velocity and are capable of changing their direction (gaining momentum from interaction with the environment);
— interacting within a specific interaction range by changing their direction of motion, in a way involving ,an effective alignment; and
— which are subject to a noise of a varying magnitude,
collective motion is bound to occur in almost all of the possible cases.

Few dozen to few thousand units. Most real life observations are at mesoscopic scale.

Leaders/ informed individual having a priori preference. 

Units that can be reached by neighbouring units belong to same cluster. Behaviour within cluster highly correlated. 

Force mapping technique - infer interaction rules from experimental data. (GPS tracking of individuals within a flock).

\subsubsection{Vicsek Model}
Self-propelled particles at constant speed, choose at discrete time steps their new direction to be an average of that of their neighbours located within unit distance. Alignment rule competes with perturbations from noise term. This noise in the communication among individuals drives a phase transition.

\subsubsection{Gregoire}
Extra attraction-repulsion term in above model, so as to prevent the dissolving of flocks in open space.

\subsubsection{Ballerini}
Starling flocks - individuals interact mostly with neighbours determined by topological rules and not metric criteria.

\subsubsection{Camperi} 
Topological models much more stable against noise and external perturbations than metric.

\subsubsection{Couzin}
change in radius of orientation - sudden transitions at school level in terms of collective behaviours. 

\subsubsection{Hemelrijk \& Hildenbrandt}
Extension on zone models to account for speed variability. Discuss shape and internal structure of schools and flocks.

\subsubsection{Grunbaum}
Spatial memory effects - distribution at group level.

\subsubsection{Romanczuk \& Schimansky-Geier}
Self-propelled particles model - only attractive and repulsive interactions. For some parameter regime, large scale collective motion without any explicit velocity-allingment mechanism.

\subsubsection{Buhl}
Locusts - field recordings and computer simpulations, to infer interactions rules.

\subsubsection{Gautrais}
Animal group motion - data gathered from individual scale. Yields model without any free parameters. 



\subsection{Collective behaviour in animal groups: theoretical models and empirical studies}

Not only in social animals but also 'selfish' birds/fish.

Information flows through the system through interactions with neighbours or chemical deposition.

Particles/magnetic moments - interact locally in space and can generate an ordered state with collective global properties.

'mechanisms leading to collective behaviour are very general' - large scale features of many order-disorder transitions don't depend on the details of local interactions but only on some general characteristics of the system. (e.g. dimensionality of space) 'universality'

Biological systems have 'motive' - finding food/nesting areas. More difficult to understand, not 1 to 1 comparison with physical systems.

\subsubsection{Agent-based flocking models}

Good for mechanisms of decision making and group formation and how behavioural attitudes translate into global features.

Assume behavioural rules at the level of the individual. 

Evolution equation specified for each agent within system, and microscopic dynamics described in terms of the social forces acting upon it. 

Mostly used for schools, flocks and swarms. 
Force experienced by individual is due to neighbour's direct interaction.
(made for schools and flocks, later used for mammal herds and vertebrate groups)
Cohesive and polarised groups formed.

    1.Move in direction of neighbours (allingment of velocities)


    2.Remain close to neighbours (Attraction)


    3.Avoid collisions. (Short range repulsion.)

\begin{equation}
\mathbf{\overrightarrow{d}_i}(t+1) = \frac{1}{n_{\text{in}}} \underset{\text{alignment}}{\sum_{j=1}^{n_{\text{in}}} w_j \mathbf{\overrightarrow{d}_j}(t) \vphantom{\frac{1}{n_{\text{in}}} \sum_{j=1}^{n_{\text{in}}}}}  
+ \frac{1}{n_{\text{in}}} \underset{\text{attraction/repulsion}}{\sum_{j=1}^{n_{\text{in}}} f_{ij} \frac{\mathbf{\overrightarrow{r}_{ij}}(t)}{\left|\mathbf{\overrightarrow{r}_{ij}}(t)\right|} \vphantom{\frac{1}{n_{\text{in}}}}} + \underset{{\text{noise}}}{\mathbf{\overrightarrow{\eta_i}}(t) \vphantom{\frac{1}{n_{\text{in}}}}},
\end{equation}

Equation for update of individual velocity
${\mathbf{\overrightarrow{d}_i}}$ - direction of motion of agent i

$\mathbf{\overrightarrow{r}_{ij}}$ - 
distance vector from agent i to j

$\mathbf{\overrightarrow{\eta_i}}$ -  stochastic noise modelling of uncertainty of decision making process. 

Alignment term = average over the directions of motion of the $\mathbf{n_{\text{in}}}$ interacting neighbours with weights $w_j$.

$f_{ij}$ force specifying how agent i attracted/repelled by agent j. Negative at short distance, positive at large.

Agent based models cannot be solved exactly. (number of individuals is large and resulting set of coupled dynamical eqs formally untreatable.) $\rightarrow$ implemented using numerical simulations. 

Neighbour preferences $\rightarrow$ angle/distance [Huth and Wissel, 1992]

COUZIN (2002) $\rightarrow$ relative extension of behavioural zones tuned to get, in 3D, groups of different shape, packing and alignment. {swarm-like aggregations (cohesion and low polarization), totoroidal milling groups (high polarization and global angularmomentum), and coherent dynamic groups (cohesion, high bpolarization and low momentum).} $\rightarrow$ strength of interaction terms analysed in Viscido, Parrish and Grunbaum (2004)

Macroscopic behaviour of materials from the microscopic behaviour of their components. 

Jadbadaie (2003) $\rightarrow$ Flocking achieved as long as network of agent interactions remains constant in time.

\subsubsection{Eulerian models}

Complementary approach to agent-based 

Deal with populations

The space represented as a lattice, the number of individuals within in the cell is followed over time. Instead of evolution of each individual as before.

Relevant variables $\rightarrow$ Density {n($\overrightarrow{x}$,t)/$V_{cell}$}, and average velocity $\overrightarrow{v}$($\overrightarrow{x}$,t) possessed by individuals in that cell.

Considering continuum limit $\rightarrow$ leads to convection-diffusion equations for the population density. (hydrodynamic eqs derived for fluids)

Eulerian models assume groups already formed, and a large enough number of members to make coarse graining meaningful. Cannot address features smaller than the cell or times shorter than that for info to reach outside the cell. $\rightarrow$ Useful for population evolution on long time scales and for patterns of large scales.


\subsubsection{Phenomelogical models}

Chemical cues $\rightarrow$ number of individuals performing a task depends on amount of perceived chemicals and so the number of individuals depositing chemical. $\rightarrow$ Ant trails

Given set of possible paths, a set of nonlinear equations can be written, describing the evolution in time of the flows of foragers along the paths

Mathematically simple.


\subsubsection{Empirical Studies}

Ballerini (2008) observational study $\rightarrow$ flocking behaviour of starlings $\rightarrow$ 3D study

Found flocks to be relatively thin and slide parallel to the ground $\rightarrow$ missed in 2D analysis.

Density and nearest neighbour distance independent of number of individuals in group. $\rightarrow$ contrary to that of small groups in previous studies.

Shape held constant proportions despite volume changes $\rightarrow$ underlying self organising mechanism. 

Angular dist of neighbours around focal individual $\rightarrow$ nearest neighbour likely to be found on sides rather than the direction of motion (as suggested to small groups of fish)

Flocks relatively sparse interacting units. Groups are denser at border than centre. 

When turning, shape remains unaltered and velocity rotates w.r.t flocks main axes. 

Interact depending on topological distance rather than metric distance and only with a fixed number of birds, not all neighbours within a fixed region of space.


\subsection{Novel Type of Phase Transition in a System of Self-Driven Particles}

Particles driven with constant absolute velocity, assumes average direction of motion of neighbouring particles with some random perturbation at each time step. $\rightarrow$ kinetic phase transition from no transport to finite net transport, through spontaneous symmetry breaking of the rotational symmetry. Transition is continuous. 

Many particle systems $\rightarrow$ complex cooperative behaviour during phase transition. equilibrium systems [Scaling, universality, renormalisation] 

Far from equilibrium [Aggregation, viscous flows, biological pattern formation]

Model used is a transport related nonequilibrium analogue of the ferromagnetic type of models, difference of being inherently dynamic. $\rightarrow$ elementary event = motion of particle between two time steps. 

Ferromagnetic interaction aligning spins $\rightarrow$ aligning direction of motion of particles (random perturbation analogy with temperature.)


\section{Week 2}

\subsection{Gregoire et al (2003)}

Vicsek model accounts well (at a qualitative level) when the organisms interact at short distances but do not have to stay together. Super-diffusive behaviour of bacteria (E. Coli [Wu, Libchaber (2000)]) swimming freely in a fluid film about as thick as the size of the bacteria. To model the the overall cohesion of a population, Vicsek's model must be adapted. In open space a flock that is, at first cohesive, will tend to disperse. (no CM possible in the zero-density limit). 

Gregoire model:
No Leader, Strongly  noisy, strictly local interactions, no confinement.
Other works that tried to adapt the Vicsek with repulsive-attractive interactions tended to do so while also including a 'flock-wide' or 'global' interaction term.

Adding a Lennard-Jones-type body $\vec{f}$ force acting between all pairs of agents within neighbour distance. To calculate the new direction of motion of each agent while including this force, the Vicsek equation becomes
\begin{equation}
    \theta_i^{t+1} = \arg \left[ \alpha \sum_{j \sim i} \vec{v}_j^t + \beta \sum_{j \sim i} \vec{f}_{ij} \right] + \eta \xi_i^t,
    \label{eq:sample_equation}
\end{equation}
where $\alpha$ and $\beta$ are the relative strength of the two 'forces'. $\vec{v}_j^t$ is the velocity vector of magnitude $\vec{v}_0$ along direction $\theta_i$ and $\xi_i^t$ is a delta-correlated white noise ($\xi$ $\in$ [-$\pi$, $\pi$]). $\eta$ is the noise strength. 

The precise form of the body force is not important, ensuring hard-core repulsion at a distance $r_c$ and equilibrium at distance $r_e$ is enough.

% Requires: \usepackage{amsmath}
\begin{equation}
    \vec{f}_{ij} = \vec{e}_{ij} 
    \begin{cases} 
        -\infty & \text{if } r_{ij} < r_c, \\
        \frac{1}{4} \frac{r_{ij} - r_e}{r_a - r_e} & \text{if } r_c < r_{ij} < r_a, \\
        1 & \text{if } r_a < r_{ij} < r_0
    \end{cases}
\end{equation}

$r_{ij}$ is the distance between two agents, $\vec{e}_{ij}$ is the unit vector along segment going from i to j.

Other forms of the noise term where also tested; the uncertainty of each agent to 'feel the force' exerted on it would change the noise term of the original equation to $N_i \eta \vec{u}_i^t$. Here $N_i$ is the number of neighbouring agents and $\vec{u}_i^t$ is a randomly orientated unit vector. 

The choice of using a particular noise term relates to how the agents transition to collective motion. Gregoire considered the latter noise term above. To ensure that each agent interacts only with the 'first layer' of neighbours, Gregoire calculated the Voronoi tessellation of the population at every time step. The interacting neighbours are then only those within distance $r_0$ which are also neighbours in the Voronoi sense. 

If body force is weak (small $\beta$) a cohesive flock cannot be maintained. Isolated agents (randomly-walking) are found for an arbitrarily large flock in an infinite space. The shape of the cohesive flock is also dependent on the value of $\beta$ ('moving droplets' and 'flying crystals')


\subsection{Ballerini et al (2007)}

Starling flocks were modelled to be relatively thin, of various sizes but of constant proportions. They slide parallel to the ground and when they are turning their orientation changed w.r.t direction of motion. Based on individual birds' wingspan, they would keep a minimum distance to each other. The birds towards the edge of the pack tend to be packed closer together than those in the centre. 

Local interactions between units without the need for centralised coordination. The tendency of each agent to imitate its neighbours (allelomimesis).

Flocks were found to be thin when viewed in 3D. When looking at the ratios of the other two dimensions to the flocks thickness, they were found to be stable and showed little dependence on the number of birds or the volume. Though, the thickness of the flock was found to be highly variable, therfore the thickness must be linearly correlated to $V^{1/3}$. Even though the volume of the flock and number of birds changed, the proportions remained roughly constant. It was also found that the plane the flock's were flying in were parallel to the ground, and the velocity perpendicular to both of these. Found to not elongate in the direction of motion. The orientation changes with respect to the velocity, though roughly constant w.r.t an absolute reference frame.

Density does not depend on the number of birds, but on the position within the flock. The nearest neighbour distance was found to relate to density as $\rho \propto r^{-3}$ (homogeneous arrangement of points). Nearest neighbour distance not dependent on size of group, though is in fish schools (Partridge et al. 1980). Flocks had higher density closer to the border, more sparse at centre. Flocks were sometimes concentrated more towards the front and back and sometimes uniformly distributed. 

Exclusion zone around each agent (to avoid collisions) found to be stable across flocks and was unaffected by the density. Average distance between the birds (starlings) ($2r_h \approx 0.38$) larger than the body length of one bird, but about the wing span. Each bird will most likely have the nearest neighbour on the plane perpendicular to velocity, and will likely not have neighbours along the direction of motion. 

Nonspherical shapes in fish schools (Cullen et al. 1965, Partridge et al. 1983) Thin shape => minimise chance of being seen by predators. Numerical simulations of fish schools (Kunz $\And$ Hemelrijk 2003, Hemelrijk $\And$ Hildenbrandt 2008) show elongated shape in direction of motion is a result of the simple local interactions themselves. Elongation and velocity correlation in fish schools (Pitcher 1980). (Partridge et al. 2008) shows fish schools become multilayered as the number of fish increases.


\subsection{Camperi et al. 2012}

Each agent interacts with a fixed number of neighbours irrespective of their distance, creating a more stable/robust structure. 








\section{Introduction} 

%%% example text follows %%%
Lorem 

\section{Methods} 

Lorem 
 
\section{Results} 

Lorem 
\section{Discussion} 

Lorem 


\section{Conclusions}
 
Lorem 


\begin{thebibliography}{}

\bibitem{ref01} A. N. Other, Title of the Book, edition, publishers, place of publication (year of publication), p. 123.   % example reference 
\cite{@article{PhysRevLett.75.1226,
  title = {Novel Type of Phase Transition in a System of Self-Driven Particles},
  author = {Vicsek, Tam\'as and Czir\'ok, Andr\'as and Ben-Jacob, Eshel and Cohen, Inon and Shochet, Ofer},
  journal = {Phys. Rev. Lett.},
  volume = {75},
  issue = {6},
  pages = {1226--1229},
  numpages = {0},
  year = {1995},
  month = {Aug},
  publisher = {American Physical Society},
  doi = {10.1103/PhysRevLett.75.1226},
  url = {https://link.aps.org/doi/10.1103/PhysRevLett.75.1226}
}
}


\end{thebibliography} 

\newpage

\section*{Error Appendix}
Lorem 


\clearpage

\onecolumngrid %Puts Summary into single column

\section*{Scientific Summary for a General Audience}



Lorem 

\end{document}